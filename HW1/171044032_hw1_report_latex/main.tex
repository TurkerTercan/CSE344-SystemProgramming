\documentclass{article}

\input{structure.tex} % Include the file specifying the document structure and custom commands

%----------------------------------------------------------------------------------------
%	ASSIGNMENT INFORMATION
%----------------------------------------------------------------------------------------

\title{CSE 344: Homework \#1} % Title of the assignment

\author{Türker Tercan\\ \texttt{turker.tercan2017@gtu.edu.tr}} % Author name and email address
    

\date{Gebze Technical University --- March 31, 2021} % University, school and/or department name(s) and a date

%----------------------------------------------------------------------------------------

\begin{document}

\maketitle % Print the title

\section*{Objective} % Unnumbered section

You are expected to write an “advanced” file search program for POSIX compatible operating
systems. Your program must be able to search for files satisfying the given criteria, and print out the
results in the form of a nicely formatted tree.

\begin{info} % Information block
	The search criteria can be any combination of the following (at least one of them must be
employed):  

• -f : filename (case insensitive), supporting the following regular expression: +

• -b : file size (in bytes)

• -t : file type (d: directory, s: socket, b: block device, c: character device f: regular file, p:

pipe, l: symbolic link)

• -p : permissions, as 9 characters (e.g. ‘rwxr-xr--’)

• -l: number of links


The program will additionally receive as mandatory parameter:

• -w: the path in which to search recursively (i.e. across all of its subtrees)


\end{info}

\section*{Example} % Numbered section

./myFind -w targetDirectoryPath -f ‘lost+file‘ -b 100 -t b


means it will search in the targetDirectoryPath path, for block device files named lost+file of
size exactly 100 bytes


\section*{Output}
In the form of nicely formatted tree
\begin{commandline}
	\begin{verbatim}
		targetDirectoryPath
|--subDirectory
|----subDirectory2
|------subDirectory3
|--------LOStttttttFile
|--subDirectory6
|----lOstFile
	\end{verbatim}
\end{commandline}

If no file satisfying the search criteria has been found, then simple output “No file found”. All error
messages are to be printed to stderr. All system calls are to be checked for failure and the user is to
be notified accordingly.


\section*{Implementation}

In order to search for files in the directories recursively, i used opendir(), readdir() and closedir() functions in the dirent.h library which supported by POSIX. 
I used the perror() function as much as possible against all possible errors. Always used syscalls for write(). 
I was not sure are we allowed to use string.h library, so I implemented my own functions which I need inside of string.h. 
The program will print usage information every time if a command line argument is missing or invalid.
For outputting the result, I implemented a simple n-child linked list for representing the directories and files.
And I implemented my program with below libraries:

• stdio.h

• stdlib.h

• unistd.h

• sys/stat.h

• dirent.h


\section*{Output Result - 1}
\includegraphics[scale=0.49]{adsız.png}

\section*{Output Result - 2}
\includegraphics[scale=0.50]{adsız1.png}

\section*{Checking For Memory Leaks}
\includegraphics[scale=0.345]{adsız2.png}
\section*{}
\includegraphics[scale=0.354]{adsız3.png}


\end{document}
